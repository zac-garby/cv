%!TEX TS-program = xelatex
%!TEX encoding = UTF-8 Unicode

\documentclass[12pt]{article}
\usepackage[margin=0.5in]{geometry}                % See geometry.pdf to learn the layout options. There are lots.
\geometry{letterpaper}                   % ... or a4paper or a5paper or ... 
\usepackage[parfill]{parskip}    % Activate to begin paragraphs with an empty line rather than an indent
\usepackage{graphicx}
\usepackage{amssymb}
\usepackage{moresize}
\usepackage{lipsum}
\usepackage{nopageno}
\usepackage{tabularx}
\usepackage{enumitem}
\usepackage{multicol}

\usepackage{fontspec,xltxtra,xunicode}
\defaultfontfeatures{Mapping=tex-text}
\setromanfont[Mapping=tex-text]{Hoefler Text}
\setsansfont[Scale=MatchLowercase,Mapping=tex-text]{Lato}
\setmonofont[Scale=MatchLowercase]{Roboto Mono}

\newcommand{\sqsep}{\quad {\tiny $\blacksquare$} \quad}
\newcommand{\gap}{\hspace*{1em}}

\setlist[itemize]{leftmargin=2.2cm, itemsep=-0.1cm}
\newlist{explist}{itemize}{4}
\setlist[explist,1]{label=\textbullet,leftmargin=2.2cm}
\setlist[explist,2]{label=$\mathbin{\vcenter{\hbox{\rule{0.5ex}{0.5ex}}}}
$,leftmargin=0.5cm,topsep=-0.2cm,itemsep=-0.2cm}
\setlist[explist,3]{label=\!, leftmargin=0.4cm, topsep=-0.25cm, itemsep=-0.2cm}
\setlist[explist,4]{label=\!, leftmargin=0.4cm, topsep=-0.25cm}

\begin{document}

\begin{center}
	{\HUGE \textbf{Zac Garby}}
	
	me@zacgarby.co.uk \sqsep
	psyzg5@nottingham.ac.uk \sqsep
%	07737 132131 \sqsep  % UNCOMMENT FOR REAL CV
	zacgarby.co.uk
\end{center}

% UNCOMMENT THESE LINES FOR REAL CV
%\begin{table}[h]
%\begin{center}
%\begin{tabularx}{\textwidth}{X r r}
%& 16 Swenson Avenue & 96 Cambridge Road \\
%& Nottingham & Dorchester \\
%& NG7 2LP & DT1 2JQ \\
%\end{tabularx}
%\end{center}
%\end{table}
%\vspace{-6em}
\vspace{-1em} % COMMENT THIS LINE FOR REAL CV

\section*{Education}

\begin{explist}
	\item[2019-present] \textbf{University of Nottingham} \\
		  MSci Computer Science, four year course.
		  {\small
		  \begin{explist}
    		  \item[] \textbf{Year 3: First (79\% subject to resits)}
		  	\begin{explist}
    		      \item Dissertation (89\%)
		      	\begin{explist}
    		          \item \textit{``Fugue, a Friendly Functional Programming Language with Holes.''}
I designed and implemented a functional programming language, named Fugue, with a novel type system based on Hindley-Milner. The language compiles to an enhanced lambda calculus and supports interactive programming to an extent using holes, and I devised a heuristic for suggesting and prioritising possible hole fills.
		          \end{explist}
		          \begin{multicols}{2}
				     \item Compilers (96\%)
				     \item Programs, Proofs \& Types (97\%)
				     \item Knowledge, Representation \& Reasoning (67\%)
				     \item Others: (72\%, 28\% subj. to resit, 78\%)
				     \end{multicols}
		        \end{explist}
    		   \item[] \textbf{Year 2: First (87\%)}
		   		\begin{explist}
    		          \item Software Engineering Group Project (\textit{89\%})
		          	  \begin{explist}
    		  		     \item \textit{``Surreal Numbers and Games.''} Using Haskell, we explored the usage of John Conway's surreal number system for general game-playing AI to produce a program that could perform well against human players even on games it had never seen. Specifically, it worked with 2-player perfect information games, and included a Haskell API for users to define their own games.
				     \end{explist}

				     \begin{multicols}{2}
				     \item Algorithms, Correctness \& Efficiency (\textit{92\%})
				     \item Operating Systems \& Concurrency (\textit{89\%})
				     \item Advanced Functional Programming (\textit{87\%})
				     \item Others (\textit{89\%, 86\%, 81\%, 80\%})
				     \end{multicols}
				 \end{explist}
    		   \item[] \textbf{Year 1: First (91\%)}
		   			\vspace*{-0.35cm}
		   			\begin{explist}
					  \begin{multicols}{2}
    		          \item Mathematics for Computer Scientists (\textit{97\%})
		              \item Programming \& Algorithms (\textit{96\%})
		              \item Databases \& Interfaces (\textit{93\%})
		              \item Others (\textit{90\%, 90\%, 88\%, 81\%, 88\%})
		              \end{multicols}
		            \end{explist}
		  \end{explist}
		  }
		  
	\item[2015-2019] \textbf{The Thomas Hardye School, Dorchester } \\
		A-Levels
		{\small
		\begin{explist}
			\item[] Mathematics: \textit{A*}; \quad Further Mathematics, Computer Science, and Physics: \textit{AAA}
		\end{explist}
		}
\end{explist}

\vspace{-1em}

\section*{Experience}

\begin{explist}
	\item[2022-present] \textit{HackSoc Nottingham,} President
	{\small \begin{explist}
		\item I am responsible for the society, including the community itself and its reputation, but also organisation and planning. I give talks and workshops, and have retained my Graphics Officer duties.
	\end{explist}}
	\item[2021-present] \textit{HackSoc Nottingham,} Lead organiser for HackNotts, our annual hackathon.
	{\small \begin{explist}
		\item I am responsible for the graphics, web development, logistics, finance, and general planning of the event.
	\end{explist}}
	\item[2021-2022] \textit{HackSoc Nottingham,} Development Secretary and Graphics Officer.
	{\small \begin{explist}
		\item I give a number of workshops and talks on tech-related topics each month.
		\item I maintain the society's website and graphics.
	\end{explist}}
	\item[2020-2021] \textit{University of Nottingham,} A Computer Science mentor.
	{\small \begin{explist}
		\item I was assigned to a small group of first-year students to help them settle in to University.
		\item I ran a number of sessions with my group to help them with their first-year modules.
	\end{explist}}
	\item[2018] \textit{National Citizen Service}, Participant.
	{\small \begin{explist}
		\item As part of a team, raised money and restored a youth centre in Dorchester.
	\end{explist}}
	\item[2017-2019] \textit{Thomas Hardye School,} Ran the Programming \& Robotics club.
	{\small \begin{explist}
		\item We taught a group of year 9 and GCSE students about programming, mainly through the context of robotics.
	\end{explist}}
	\item[2017-2019] \textit{Thomas Hardye School,} Volunteered at a number of STEM days.
	{\small \begin{explist}
		\item We ran half-day sessions teaching middle school students about programming and simple robotics using LEGO Mindstorm.
	\end{explist}}
\end{explist}

\section*{Skills \& Interests}
\begin{itemize}
	\item Extensive experience in Haskell (>6 years), Python (>10 years), C, Go, JavaScript, Agda, and \LaTeX. Also Rust, Java, various LISPs, and numerous domains specific languages.
	\item Strong interest in many areas related to programming language theory, including type theory, compiler design/implementation, and interactivity in programming languages.
	\item Experience with systems programming, scientific computing in Python and MATLAB, full-stack web development, networking, multimedia (image processing, audio processing/synthesis, game development), and the design and implementation of programming language compilers.
	\item Strong interest in hackathons, both as an attendee and as an organiser.
	\item Interested in mathematics, especially where it overlaps with Computer Science, but also mathematical puzzles and games for their own sake.
	\item I enjoy playing, listening to, and creating music; I play the guitar and the piano. I also enjoy reading, and I am a long-time member of my University's Medival Combat Society.
\end{itemize}

\section*{Awards \& Achievements}
\begin{itemize}
	\item[2022] \textit{Computer Science, University of Nottingham,} Best Individual Year Three Dissertation prize.
	\item[2022] \textit{Computer Science, University of Nottingham,} High Achiever's Award (top 5 in my year.)
	\item[2021] \textit{AstonHack 2021,} First place for my project, ``Network over Rube Goldberg Machine''.
	\item[2021] \textit{Computer Science, University of Nottingham,} High Achiever's Award (top 5 in my year.)
	\item[2020] \textit{HackNotts 2020,} Sponsored prize for my project, ``The Haskelltron 2000''.
	\item[2020] \textit{Computer Science, University of Nottingham,} High Achiever's Award (top 5\% in my year.)
	\item[2019] \textit{Computer Science, University of Nottingham,} Silver Scholarship (a 25\% tuition fee rebate each of my four years at University, so long as I achieve 80\% in each.)
	\item[2019] \textit{Thomas Hardye School,} Selected by my school to create an interactive exhibit for the local community's ``50th Anniversary of the Moon Landing'' event.
	\item[2019] \textit{Thomas Hardye School,} Received my school's first ever Computer Science subject award.
	\item[2018] \textit{United Kingdom Mathematics Trust,} Silver award in their Senior Mathematical Challenge.
	\item[2015] \textit{Bournemouth University,} Second place out of hundreds of entries in a programming competition.
\end{itemize}

\section*{References}
Available upon your request.

\end{document}  